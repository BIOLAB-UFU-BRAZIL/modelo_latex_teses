%---------------------------------------------------------------
%	UNIVERSIDADE FEDERAL DE UBERLÂNDIA
%	Faculdade de Engenharia Elétrica
%	Programa de Pós-Graduação em Engenharia Elétrica
%	Programa de Pós-Graduação em Engenharia Biomédica
%	Laboratório de Engenharia Biomédica
%	Dissertação de Mestrado	
%	Capítulo 1: Introdução
%---------------------------------------------------------------

\chapter{Introdução}
\label{cap:introducao}
O sistema motor é quem nos garante a capacidade de movimentar o nosso corpo livremente através da ação coordenada dos nossos músculos. Tal capacidade pode ser vista como uma característica fundamental à vida, possibilitando a interação com o meio ambiente e outros seres além de transformar em ações concretas o nosso pensamento, imaginação e criatividade \cite{Kandel2013}. Para tanto, uma complexa rede de conexões neurais e de diferentes estruturas corticais e sub-corticais atuam como os responsáveis por regular o funcionamento do sistema neuromuscular. Estas conexões geram um programa motor que pode se ajustar de acordo com a tarefa a ser desempenhada e as informações sensoriais recolhidas de todo o nosso corpo. Os processos que regem o controle motor são, portanto, complexos e as estratégias utilizadas pelo sistema nervoso ainda são objeto de pesquisas e debates científicos \cite{Kandel2013, Lent2010, Guyton2006}.

\section{Motivação}

A Figura \ref{fig:figura1} está abaixo

\begin{figure}[!htbp]
\begin{center} 
\includegraphics[width = 0.2\textwidth]{img/logo_ufu.pdf}
\caption[Diagrama de blocos do sistema motor]{Diagrama de blocos do sistema motor \cite{Kandel2013}.}
\label{fig:figura1}
\end{center}
\end{figure}


\section{Objetivos}
